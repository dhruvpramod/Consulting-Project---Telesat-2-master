%--------------------------% 
% PREAMBLE 1 - DO NOT EDIT %
%--------------------------% 
\input{Admin/Proposal/ProposalPreamble1.tex}

%-----------------------------% 
% PROJECT-SPECIFIC PARAMETERS %
%-----------------------------% 
%\newcommand{\thedraft}{(DRAFT)}
\newcommand{\thedraft}{}

\newcommand{\titlestring}{Flight Route Predictive Analytics Model for Telesat LEO}
\newcommand{\theprojectnumber}{4376E-F19-TELESAT-002}
\newcommand{\theclient}{Dr. Maryam Haghighi}
\newcommand{\theshortclient}{Dr. Haghighi}
\newcommand{\theclientrole}{Manager \\ Satellite Communications Analytics}
\newcommand{\theclientaddress}{160 Elgin Street, Suite 2100 \\ Ottawa, Ontario  K2P 2P7 \\ Canada} % Project Title and Project Number

%--------------------------% 
% PREAMBLE 2 - DO NOT EDIT %
%--------------------------%
\title{\textcolor{acolor}{\LARGE{RE: \titlestring \thedraft}}}

%\author{\ \\ \normalsize{\thestudents Patrick Boily} \\ \small{Department of Mathematics and Statistics, University of Ottawa}  \\ \ \\ \today }


\renewcommand{\sectionmark}[1]{\markboth{\textsc{#1}}{}}
\newcommand{\newl}{\newline\newline}

\begin{document}
\begingroup
\let\center\flushleft
\let\endcenter\endflushleft
\noindent \theclient \hfill Project Number: \theprojectnumber \\ 
\theclientrole \hfill  \\
\theclientaddress
\begin{flushleft}\today\end{flushleft}
\maketitle
\thispagestyle{fancy}
\lhead{}
\chead{\includegraphics[height=35pt]{Images/uOttawa}\hfill \includegraphics[height=35pt]{Images/dms}}
\rhead{}
\lfoot{\ }
\cfoot{\footnotesize{STEM Complex, room 541 $-$ 150 Louis-Pasteur Pvt., 
Ottawa, ON, Canada K1N 6N5 $-$ \newhref{mailto:pboily@uottawa.ca}{pboily@uottawa.ca}}}
\endgroup




\noindent \theshortclient, \newl 
Thank you for providing a course-based consulting experience for the students of MAT 4376G Topics in Statistics (Introduction to Quantitative Consulting), offered by the Department of Mathematics and Statistics at the University of Ottawa during the 2019 Fall Term. \newl 
This project will help them bridge the gap between theory and applications of mathematical, statistical, and analytical methods. Under my supervision, the students will work in small teams and treat this mock project as a real-world consulting opportunity, \textbf{with the caveat that no contractual obligation exists between your organization and the University}. \newl In particular, while the team will strive to provide useful and actionable insights into your problem, the University makes no guarantee that the project will be successful and, consequently, will not charge a fee for services rendered. Telesat, on the other hand, will retain all IP generated \textit{via} this project related to satellite communication applications.    
%--------------------------% 
% PREAMBLE 3 - DO NOT EDIT %
%--------------------------%
\tableofcontents
\newpage\noindent

\pagestyle{fancy}
\lhead[]{}
\rhead[]{}
\chead{\textsc{Proposal (\titlestring)}}
\rfoot[\footnotesize{\thepage}]{\footnotesize{Project Number: \theprojectnumber \thedraft}}
\cfoot{}
\lfoot[\footnotesize{Project Number: \theprojectnumber  \thedraft}]{\footnotesize{\thepage}}


%-------------------%
% START OF PROPOSAL %
%-------------------%
\section{Background}
Telesat is a global leader in satellite operation, providing reliable and secure satellite-delivered communications solutions worldwide. They provide high value expertise and support to industry participants on a global basis. Telesat owns 16 GEO satellites, the Canadian payload on ViaSat-1 and one Phase 1 LEO satellite which is the start of Telesat’s planned advanced global LEO satellite constellation that will offer ultra-low latency, extremely high throughput, affordable broadband services.\par
Telesat is launching a state-of-the-art satellite constellation of highly advanced satellites in low-earth-orbit (LEO) that will seamlessly integrate with terrestrial networks. The global network will deliver fiber quality throughput anywhere on earth. This is a highly flexible system that dynamically allocates capacity where there’s demand, thus maximizing system efficiency. This is a future-proof solution to backhaul cellular traffic, and to provide high-speed broadband access to planes, ships and remote enterprise and government users. Capacity will be available as layer-2 connectivity for service providers to connect their end users.
\section{Objectives and Scope}
\begin{itemize}
\item Provide advanced knowledge of the flight routes over the European Corridor, including the latitude and longitude position of an aircraft that is planned to receive service from the satellites, through forecasts and time-weighted probability distribution for each circular coverage area
\item Provide visualization representing flight routes and density, including heat maps/flame charts for the circular coverage area to help with easily interpretation and use of the analysis provided
\end{itemize}
\section{Methodology}
The tasks to be undertaken for analysis are outlined below:
\begin{itemize}
    \item \textit{Data Collection} -- we will be getting data from Telesat and other online sources, and possibly merging multiple data sets to improve predictions;
    \item \textit{Data Cleaning and Formatting} -- we will review the data on flight history, cleanse and format it to meet the requirements of our alagorithms;
    \item \textit{Analysis and Modeling of Data} -- we will analyze the data using time-series algorithms and machine learning methods to predict the location of each flight. Our analysis will account for statistical dependencies of each coverage area and the final model will be able to predict the density for each coverage region;
    \item \textit{Visualisation of Results} -- we will provide visualisations of our analysis representing flight routes and density, including heat maps/flame charts for the circular regions, and
    \item \textit{Documentation of Findings} -- we Will create a short report detailing the results obtained above and provide a dashboard.
\end{itemize}
%- suggest a series of steps / methods that you will follow (see samples); the idea is to show the client that you have already started thinking about their problem\\
%- add caveat that the data will ultimately be driving what method is used 
\section{Milestones and Deliverables}
The following deliverables will be provided to Telesat:
\begin{itemize}
    \item monthly progress reports providing updates on completed tasks;
    \item code for time-series analysis and other analysis of data (R, Python, or other, TBD);
    \item final report containing findings and conclusions, and
    \item dashboard to report findings of interest.
\end{itemize}
\section{Schedules and Assumptions}
\begin{itemize}
    \item \textbf{Progress Reports} - Oct 31, Nov 30
    \item \textbf{Final Report} - Dec 21
    \item \textbf{Dashboard} - Dec 21
\end{itemize}
Provided that this proposal is agreed to by October 10, 2019, we expect the project to be completed by December 21. Unexpected circumstances may require adjustments to this schedule. Should there be a change in scope and/or level of effort required, the timeline will be modified accordingly through agreement with Telesat. If we are unable to complete a task in time, we will inform Telesat as soon as possible in order to re-organize the project’s priorities.

\section{Resources and Costs}
The team assigned to this project consists of Dr.\ Patrick Boily (team lead), Vatya Kashore, Smit Patel, and Dhruv Pramod (undergraduate students in statistics with strong analytical and problem-solving skills). The work will be completed using open-source software (R, Python, etc.). We do not anticipate any need to travel for this project, although we suggest setting up a few meetings with Telesat to clarify questions relating to domain expertise, modeling approaches, and the available data. 
\newl As this project is a mandatory component for the course \textit{MAT 4376E Introduction to Quantitative Consulting}, the client will not be charged for any work completed -- the total projected cost is thus nil. Note that this arrangement is only applicable within the context of the course and not necessarily applicable to extensions of this work or future work with Telesat. 
\section{Invoicing}
A mock invoice (with a total cost of 0\$) will be sent upon completion of the project, for your records. 


%\begin{center}
%    \rule{0.5\textwidth}{.4pt}
%\end{center}
\newpage\noindent We look forward to working with your organization in this matter, \newl 
Regards, \newl
\includegraphics[height=60pt]{Images/Signature_PB.png} \newl
Patrick Boily, Ph.D. \hfill 541 STEM Complex\\
Department of Mathematics and Statistics \hfill 150 Louis-Pasteur Private\\ 
University of Ottawa \hfill Ottawa, ON, K1S 5B6 \newl
Phone: 613-562-5800 ext. 3526\\ 
Email: \newhref{mailto:pboily@uottawa.ca}{pboily@uottawa.ca}
\begin{center}
    \rule{0.5\textwidth}{.4pt}
\end{center}
\appendix

\section{Suggested Workplan}



\begin{center}
\begin{tabular}{|c|c|c|}
\hline
Tasks & Expected Date of Completion & Expected Effort \\
\hline
\hline
Data Collection & October 20 & 5hr \\
\hline
Data Cleaning/Formatting & October 20 & 25hrs \\
\hline
Preliminary Analysis of Data & November 1 & 30hrs \\
\hline
Create Flight Route Forecast & November 10 & 40hrs \\
\hline
Create Time-Varying Probability Distribution & November 20 & 30hrs \\
\hline
Visualisation of Analysis & November 25 & 20hrs \\
\hline
Create Heat Maps/Flame Charts & November 30 & 10hrs \\
\hline
Create Dashboard with Findings & December 10 & 10hrs \\
\hline
Documentation of Findings & December 21 & 10hrs \\
\hline
\end{tabular}
\end{center}
\noindent
Based on the estimated time and effort, the total cost associated with this project would would be \$14,250 (+ HST), at an hourly rate of \$60/hour for students (150 hours) and \$175/hour for myself (30 hours). Since this project is course work for MAT 4376E, the cost to Telesat is \$0.

\section{Credentials}
\textbf{Patrick Boily, Ph.D.} — Patrick is a graduate from the University of Ottawa. He obtained his Ph.D. in Mathematics in 2006. He has taught over 40 courses at Universities in the Ottawa area since 1999, and worked on a number of projects as a federal public servant from 2008 to 2012, including the award-winning Canadian Vehicle Use Study. He started and managed Carleton University’s \textit{Centre for Quantitative Analysis and Decision Support} from 2012 to 2019. He is now a professor in the University of Ottawa’s Department of Mathematics and Statistics. \par Patrick’s academic interests reside in the application of mathematics and statistics to evidence-based decision support. He has provided consulting services to numerous entities over the years, including United Way, the Public Health Agency of Canada, the Canadian Air Transport Security Authority, the Royal Canadian Mounted Police, Transport Canada, the Nuclear Waste Management Organization, the Privy Council Office, the Canada School of Public Service, and Correctional Services Canada. \par He has extensive experience in operations research, data science and predictive analytics, stochastic modeling, and simulations – managing and being involved in numerous projects in these subject areas from inception to completion. He also leads various workshops on data science and statistical analysis.
\begin{center}
    \rule{0.5\textwidth}{.4pt}
\end{center}

\noindent
\textbf{Vatya Kishore} — Undergraduate student at the University of Ottawa. Currently enrolled as a fourth year student in the Joint Honours B.Sc. in Mathematics and Economics program. He has honed analytical skills and gained basic proficiency in Stata and R through Economics and Mathematics university course work. He also gained intermediate proficiency in Java and Python from course work throughout high school and a couple of courses in university, and has been awarded with a certificate of excellence for participating in the Canadian Computing Competition.\par He has built strong professional communications skills while working at Enterprise Holdings, Hudson's Bay and Ontario Consumers Home Services (OCHS), through daily communications with employees and customers. Improved leadership and organisational skills while training some of the new employees at OCHS and as a team lead on multiple occasions at Enterprise Holdings. He has worked as a volunteer at the Residents’ Association University of Ottawa (RAUO) as a Floor Representative where he helped coordinate multiple RAUO events and RAUO elections. He has also volunteered as a peer tutor in high school.
\begin{center}
    \rule{0.5\textwidth}{.4pt}
\end{center}

\noindent
\textbf{Smit Patel} - Smit is a university student currently working towards a Bachelor of Sciences in Mathematics with a Minor in Computer Science at the University of Ottawa. While doing so, he has been able to gain experience at many companies both in the private and public sector. He has worked at the Apple store as a Technical Specialist, Shared Services Canada as a Technical Intern, and is currently working for Canada Border Services Agency as a Junior Programmer. These jobs all require problem solving, communication and teamwork, helping Smit to not only be able to work within a team setting and solve problems, but also manage others. \par Smit has also been a mentor for his high school robotics team, in which he participated in when he was in high school, and so his passion for machine learning and automation blossomed. Smit’s academic interests reside in the application of mathematics to progress machine learning as well as the development of automation to make workflows more efficient. His experience includes programming with Java and Python, statistic analysis with R, and creating automation scripts using XML’s.
\begin{center}
    \rule{0.5\textwidth}{.4pt}
\end{center}

\noindent
\textbf{Dhruv Pramod} – Dhruv is currently an undergraduate student at the University of Ottawa, pursuing a degree in the Statistics field. He has attained a wealth of quantitative experience, performing cluster analysis and using regression analysis for course work in mathematics and statistics. He is a proficient user of R statistical software and has working knowledge of SQL as he was involved in the back- end development of a co-op database. He has honed his communication, leadership, and problem resolution skills through holding positions like Team Lead at Canada’s Wonderland and Market Research Representative at Elemental Data Collection. \par Dhruv has a keen interest for predictive analytics, time series analysis, and risk management which stems from his love for Actuarial Science. He has a burning passion for applying mathematics to solve real world problems and aims to do so by leveraging his creative problem solving and analytical skills. 
%%%%%
% ADD YOUR SHORT BIOS HERE
%%%%%
\end{document}
